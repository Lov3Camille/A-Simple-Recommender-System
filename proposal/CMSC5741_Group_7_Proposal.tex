%%
%% This is file `sample-authordraft.tex',
%% generated with the docstrip utility.
%%
%% The original source files were:
%%
%% samples.dtx  (with options: `authordraft')
%% 
%% IMPORTANT NOTICE:
%% 
%% For the copyright see the source file.
%% 
%% Any modified versions of this file must be renamed
%% with new filenames distinct from sample-authordraft.tex.
%% 
%% For distribution of the original source see the terms
%% for copying and modification in the file samples.dtx.
%% 
%% This generated file may be distributed as long as the
%% original source files, as listed above, are part of the
%% same distribution. (The sources need not necessarily be
%% in the same archive or directory.)
%%
%% The first command in your LaTeX source must be the \documentclass command.
\documentclass[sigconf,authordraft]{acmart}
%% NOTE that a single column version may be required for 
%% submission and peer review. This can be done by changing
%% the \doucmentclass[...]{acmart} in this template to 
%% \documentclass[manuscript,screen,review]{acmart}
%% 
%% To ensure 100% compatibility, please check the white list of
%% approved LaTeX packages to be used with the Master Article Template at
%% https://www.acm.org/publications/taps/whitelist-of-latex-packages 
%% before creating your document. The white list page provides 
%% information on how to submit additional LaTeX packages for 
%% review and adoption.
%% Fonts used in the template cannot be substituted; margin 
%% adjustments are not allowed.
%%
%% \BibTeX command to typeset BibTeX logo in the docs
\AtBeginDocument{%
  \providecommand\BibTeX{{%
    \normalfont B\kern-0.5em{\scshape i\kern-0.25em b}\kern-0.8em\TeX}}}

%% Rights management information.  This information is sent to you
%% when you complete the rights form.  These commands have SAMPLE
%% values in them; it is your responsibility as an author to replace
%% the commands and values with those provided to you when you
%% complete the rights form.
%\setcopyright{acmcopyright}
%\copyrightyear{2018}
%\acmYear{2018}
%\acmDOI{10.1145/1122445.1122456}

%% These commands are for a PROCEEDINGS abstract or paper.
%%\acmConference[Woodstock '18]{Woodstock '18: ACM Symposium on Neural
 % Gaze Detection}{June 03--05, 2018}{Woodstock, NY}
%\acmBooktitle{Woodstock '18: ACM Symposium on Neural Gaze Detection,
 % June 03--05, 2018, Woodstock, NY}
%\acmPrice{15.00}
%\acmISBN{978-1-4503-XXXX-X/18/06}


%%
%% Submission ID.
%% Use this when submitting an article to a sponsored event. You'll
%% receive a unique submission ID from the organizers
%% of the event, and this ID should be used as the parameter to this command.
%%\acmSubmissionID{123-A56-BU3}

%%
%% The majority of ACM publications use numbered citations and
%% references.  The command \citestyle{authoryear} switches to the
%% "author year" style.
%%
%% If you are preparing content for an event
%% sponsored by ACM SIGGRAPH, you must use the "author year" style of
%% citations and references.
%% Uncommenting
%% the next command will enable that style.
%%\citestyle{acmauthoryear}

%%
%% end of the preamble, start of the body of the document source.
\begin{document}

%%
%% The "title" command has an optional parameter,
%% allowing the author to define a "short title" to be used in page headers.
\title{Frequent Item Sets Mining for Recommender Systems}
\subtitle{CMSC 5741 Group 7}

%%
%% The "author" command and its associated commands are used to define
%% the authors and their affiliations.
%% Of note is the shared affiliation of the first two authors, and the
%% "authornote" and "authornotemark" commands
%% used to denote shared contribution to the research.
\author{Ziwen LU}
\email{1155155161@link.cuhk.edu.hk}
\affiliation{%
  \institution{Department of Information Engineering}
  \school{The Chinese University of Hong Kong}
}

\author{Yan WU}
\email{1155148594@link.cuhk.edu.hk}
\affiliation{%
  \institution{Department of Information Engineering}
  \school{The Chinese University of Hong Kong}
}

\author{Yaling ZHANG}
\email{1155147233@link.cuhk.edu.hk}
\affiliation{%
  \institution{Department of Information Engineering}
  \school{The Chinese University of Hong Kong}
}

\author{Bowen FAN}
\email{1155155953@link.cuhk.edu.hk}
\affiliation{%
  \institution{Department of Information Engineering}
  \school{The Chinese University of Hong Kong}
}


%%
%% By default, the full list of authors will be used in the page
%% headers. Often, this list is too long, and will overlap
%% other information printed in the page headers. This command allows
%% the author to define a more concise list
%% of authors' names for this purpose.

%%
%% The code below is generated by the tool at http://dl.acm.org/ccs.cfm.
%% Please copy and paste the code instead of the example below.
%%
\begin{CCSXML}
<ccs2012>
 <concept>
  <concept_id>10010520.10010553.10010562</concept_id>
  <concept_desc>Computer systems organization~Embedded systems</concept_desc>
  <concept_significance>500</concept_significance>
 </concept>
 <concept>
  <concept_id>10010520.10010575.10010755</concept_id>
  <concept_desc>Computer systems organization~Redundancy</concept_desc>
  <concept_significance>300</concept_significance>
 </concept>
 <concept>
  <concept_id>10010520.10010553.10010554</concept_id>
  <concept_desc>Computer systems organization~Robotics</concept_desc>
  <concept_significance>100</concept_significance>
 </concept>
 <concept>
  <concept_id>10003033.10003083.10003095</concept_id>
  <concept_desc>Networks~Network reliability</concept_desc>
  <concept_significance>100</concept_significance>
 </concept>
</ccs2012>
\end{CCSXML}

%%
%% Keywords. The author(s) should pick words that accurately describe
%% the work being presented. Separate the keywords with commas..
\maketitle

\section{Introduction}

The retail industry took a distinguished turn with the flourish of online shopping. With the speed and convenience of online retail, it has become easier for consumers to get what they want when they want it. Moreover, due to the influence of COVID-19, people are more inclined to shop online recently. However, the online shopping industry can be cutthroat. This is why understanding online shopping statistics is more important now than ever to get ahead of the competition. 

\subsection{Motivation}

Customers always have great expectations from brands they are interested in. In this case, providing customized product recommendation to different customers will increase the retailers’ competitiveness. Therefore effective recommendation system which filters a large scale of information becomes necessary. However, the ongoing rapid expansion of online shopping and the diversity of customers’ interests makes it difficult to conduct recommendation. To further illustrate such difficulty, firstly, the number of digital buyers worldwide keeps climbing every year. In 2019, an estimated 1.92 billion people purchased goods or services online. During the same year, e-retail sales surpassed 3.5 trillion U.S. dollars worldwide, and according to the latest calculations, e-commerce growth will accelerate even further in the future \cite{online-shopping}. Secondly, customers’ shopping behavior can be affected by different factors such as regions, personal habits, global events, etc. For example, In a world-wide statistics, the top online categories for purchasing are fashion (61$\%$), travel (59$\%$), books and music (49$\%$), IT (47$\%$), and events (45$\%$), but in the Asia Pacific, the most popular online industries are packed groceries (40$\%$), home care (37$\%$), fresh groceries (35$\%$), and video gaming (30$\%$) \cite{e-commerce}. Moreover, retail platforms have undergone an unprecedented global traffic increase between January 2019 and June 2020, surpassing even holiday season traffic peaks. Overall, retail websites generated almost 22 billion visits in June 2020, up from 16.07 billion global visits in January 2020. This is of course due to the COVID-19 which has forced millions of people to stay at home in order to stop the spread of the virus \cite{covid-19}. 
Considering all these factors above, it does make sense to analyze a large scale of data set and extract inspiring advice for nowadays’ recommendation systems.

\subsection{Objectives}

Many other academic researches tend to focus on improving the efficiency or capability of recommender systems. Based on several topics of CMSC5741, this project aims to present more dimensions in regards to item recommendations. For instance, what brands do they recommend to shoppers, at what range of prices’ items shoppers are inclined to buy, when recommendations can be given to shoppers according to their shopping preference, what kind of similar items should be recommended, etc. 

\subsection{Deliverables}
This project will provide following items:\\
1.  A prototype system that recommend goods based on the frequency of items purchased by similar users\\
2.  Clusters of similar users according to their purchase behavior, if possible

\subsection{Relevance to the Course}
This project will be relevant to the following topics taught in our course:\\
1. MapReduce:Upon the purchase data(i.e. product ID )is achieved about each user, then giving the product ID to each user to make the user becomes a basket.\\
2. Frequent Itemsets: Analyze all commodity purchase data and find some goods that the support threshold is over X. In those frequent itemsets, other goods could be the recommend goods for customers.\\
3. Clustering: Obtain the clusters of similar users,if possible.\\
4. 4V Feature: Enough volume of datasets will be included.

\section{Related Work}
As for the methods of dealing with selected frequent itemsets, \cite{10.1145/1141277.1141410} proposed a method on decision trees to further enhance the accuracy. Its' approach mainly relies on the relationship between customer purchase history and optimal recommendation goods.\cite{10.1145/1639714.1639799} has suggested a method of relevant set correlation to do clustering in recommender system. The similarity of two items depends on the number of common neighbours they have.


\section{Methodology}

\subsection{Dataset}
This project uses E-commerce behavior data from multi-category store available on\cite{market.org} as datasets.The datasets contain 285 million users’ purchasing events from a large multi-category online store for a duration of 7 months, dated from October 2019 to April 2020. Each dataset is shown in a CSV formart.The statistics of the datasets is summarized in Table 1.
\begin{table}[H]
    \centering
    \caption{The Statistics of Datasets}
    \label{Table 1}
    \begin{tabular}{lll}
\hline
Dataset              & Size         & Records                \\
\hline
2019-Oct             & 5.27GB       & 42,448,765              \\
2019-Nov             & 8.38GB       & 67,501,980              \\
2019-Dec             & 8.71GB       & 67,542,879              \\
2020-Jan             & 7.24GB       & 55,967,042              \\             
2020-Feb             & 7.14GB       & 55,318,566              \\           
2020-Mar             & 7.28GB       & 56,341,242              \\        
2020-Apr             & 8.62GB       & 66,589,269              \\
\hline
Total Size of Datasets & 52.64GB                             \\
Total No. of Records & 411,709,743                          
\\
\hline
\end{tabular}
\end{table}


\subsection{Overview of Main Techniques and Algorithms}
To handle such big volume of data, this project will first extract and sort topic relevant information with cloud MapReduce. The expected output will be UserId, ProductID, Product Brand and Product Categories.With such output, this project treats each user as a basket and applies A-priori algorithm to find the frequent itemset from each user. The frequent itemsets give us information about what products are usually bought together and therefore can be used to recommend related products.\\

\noindent To be more clearer, the process can be shown below:
\begin{itemize}
\item[$\bigstar$] Step1 Pre-processing: conduct MapReduce to select information needed.
\item[$\bigstar$] Step2 Implementation: use frequent itemsets algorithm\\
(Apriori/SON/PCY) to choose recommended goods.
\item[$\bigstar$] Step3 Improvement: add decision trees algorithm to reach more accurate result.
\item[$\bigstar$] Step4 Bonus: make full use of datasets to analyze customer behaviour of purchase and create visualized analysis.
\end{itemize}
\subsection{Evaluation Methods}
Instead of showing all the frequent itemsets at the output, this project will inquire a user what products he/she is interested in. Based on users' responses, the system can scan through frequent itemsets to find related products for recommendation. The users are encouraged to input more than one product, as the recommendation accuracy can be improved with multiple inputs. The option of implementing a decision tree when scanning through the frequent itemsets for a more reliable result is considered.
\section{Expectation}
Table 2 shows the expected project milestone.
\begin{table}[H]
    \centering
    \caption{Project Planner}
    \label{Table 2}
\begin{tabular}{ll}
\hline
Time                  & To do list                                      \\
\hline
2020/10/27-2020/11/3  & Search Dataset,Determine Topic                \\
                      & Write Proposal                                \\
2020/11/4-2020/11/10  & Set up Environment, Define Existing            \\
                      & Algorithm                                       \\
2020/11/11-2020/11/17 & Implement in local environment, Debug              \\
2020/11/18-2020/11/24 & Implement on Hadoop platforms, Debug                  \\
2020/11/25-2020/12/2  & Summarize project codes, Write Report and slides          \\
                      & Record demo video                             \\
\hline
\end{tabular}
\end{table}
\noindent Here are some immatured ideas about what to explore further in this project:\\
1. Do similar user clustering, so that the interface can present both recommended goods and how many similar users are purchasing this kind of products.\\
2. Analyze customer purchase behaviours via the information of purchasing events, price and brand preference.
\medskip

%Sets the bibliography style to UNSRT and imports the 
%bibliography file "samples.bib".
\bibliographystyle{unsrt}
\bibliography{CMSC5741_Reference}

\end{document}
\endinput
%%
%% End of file `sample-authordraft.tex'.
